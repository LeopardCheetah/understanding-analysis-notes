\documentclass[../analysis_notes.tex]{subfiles}

\begin{document}

\subsubsection{Notes}

$\mathbb{N} \Longrightarrow {1, 2, \dots}$. \\
$\mathbb{Z} \Longrightarrow {\dots, -2, -1, 0, 1, 2, \dots}$. \\
$\mathbb{Q} \Longrightarrow \left\{\frac{p}{q} \text{ where } p, q \in \mathbb{Z}, q \not = 0\right\}.$ \\


\textbf{Field.}    
A field is a set of numbers where two operations ($+$, $\times$ - called addition and multiplication) are well-defined (closed), commutative, associative, and distributive (in some sense). Also, additive/multiplicative inverses exist for each element, and there exists two distinct identities; the additive identity (0) and the mulitplicative identity (1). \\

Note that $\mathbb{Q}$ is ordered: for any $r, s \in \mathbb{Q}$, either $r > s$, $r = s$, or $r < s$.        
In some sense, $\mathbb{Q}$ is ``dense'' (kinda) since it spans the whole number line and for any $r, s \in \mathbb{Q}$ such that $r < s$, you can always find some $t \in \mathbb{Q}$ such that $r < t < s$.

\end{document}
