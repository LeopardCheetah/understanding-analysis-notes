\documentclass[../analysis_sols.tex]{subfiles}

\begin{document}


\begin{exercise}
    \indent
    (a) Prove that \( \sqrt{3} \) is irrational. Does a similar argument work to show \( \sqrt{6} \) is irrational?

    (b) Where does the proof of Theorem 1.1.1\footnotemark{} break down if we try to use it to prove \( \sqrt{4} \) is irrational?
\end{exercise}
\footnotetext{maybe add a hyper link to the thm 1.1.1 in the actual textbook?}
 
% sol 1.2.1 -- TODO
\begin{solution}
    hmmm
\end{solution}


% ex 1.2.2
\begin{exercise}
    Show that there is no rational number \( r \) satisfying \( 2^{r} = 3 \).
\end{exercise}

% sol 1.2.2 -- TODO
\begin{solution}
    solution to 1.2.2.
\end{solution}




% ex 1.2.3 -- TODO -- finish + fix
\begin{exercise}
    Decide which of the following represent true statements about the nature of sets. For any that are false, provide a specific example where the statement in question does not hold.

    (a) If \( A_{1} \subset A_{2} \) then \( x \).

    (b) if abc, then d.     

    (c) test
\end{exercise}

% sol 1.2.3 -- TODO
\begin{solution}
    and their corresponding solutions.
\end{solution}


\end{document}
