\documentclass[../analysis_notes.tex]{subfiles}

\begin{document}

\subsubsection{Notes}

Preliminaries:    

Sets
\begin{itemize}
    \addtolength\itemsep{4mm}
    \addtolength\parskip{-8mm}
    \item \( A \) and $B$ are disjoint if $A \cap B = \emptyset$.
    \item $A^{\text{c}}$ usually represents ${x \in \mathbb{R}, x \not \in A}$.
\end{itemize}

Functions:
\begin{itemize}
    \addtolength\itemsep{4mm}
    \addtolength\parskip{-8mm}
    \item $f: A \rightarrow B$
    \item Domain: set $A$. Range: subset of $B$ which every input from $A$ ``hits'' (maps to).
    \item so $Range(f)$ is not necessarily equal to $B$.
\end{itemize}

(8) - Proof of the triangle inequality, introduction of the absolute value function. 

$\rightarrow$ iff (if and only if) $\Rightarrow$ Must prove both sides of the equation. \\ 
Prove $p$ iff $q$ is the same as saying prove both (if $p$, $q$) and (if $q$, $p$).

\hr   
\newpage 



\subsubsection{Exercises}

\begin{exercise}
    \indent
    (a) Prove that \( \sqrt{3} \) is irrational. Does a similar argument work to show \( \sqrt{6} \) is irrational?

    (b) Where does the proof of Theorem 1.1.1 break down if we try to use it to prove \( \sqrt{4} \) is irrational?
\end{exercise}


% sol 1.2.1 
\begin{solution}
    (a) Essentially, we follow the proof for Theorem 1.1.1. 

    Assume there exists some rational \( \frac{p}{q} \) ($p \in \mathbb{Z}$, $q \in \mathbb{N}$) with $\gcd(p, q) = 1$ such that $\left(\frac{p}{q}\right)^2 = 3$. Simplifying, we get that $p^2 = 3q^2$. Since the right hand side (RHS) of our equation is divisible by 3, the left hand side (LHS), $p^2$, must also be divisible by 3, implying $p$ must be divisible by 3. As such, we can represent $p = 3k$ for some $k \in \mathbb{Z}$. Substituting this into our original equation, we find that $3k^2 = q^2$, which in turn implies $q$ is divisible by 3 (see the logic above). \\
    Yet if both $p$ and $q$ are divisible by 3, then our original fraction $\left(\frac{p}{q}\right)$ was not in fact in its most simplified form; namely, $\gcd(p, q) = 3$! As such, our original assumption was wrong, meaning there exists \textbf{no} such rational $r = \frac{p}{q}$ such that $r^2 = 3$.
    
    A similar argument does in fact work to prove that there exists no rational $r$ such that $r^2 = 6$.

    \[\] % phantom

    (b) The proof breaks down when we look at the divisibility of both sides. Namely, following the same steps at before, we arrive at the equation $p^2 = 4q^2$. Here, following the same logic as before, the 4 on the RHS implies $p$ is even, meaning that $p = 2k$ for some $k$. As such, $p^2 = 4q^2 \rightarrow (2k)^2 = 4q^2 \rightarrow k^2 = q^2$. Previously, we would now be able to claim something about $q$ (e.g. $q$ must be divisible by 3). Yet currently, the LHS of the equation gives us no information about $q$. As such, there is no logical contradiction (or infinite descent argument) for us to exploit to prove the irrationality of $\sqrt{4}$.
\end{solution}


% ex 1.2.2
\begin{exercise}
    Show that there is no rational number \( r \) satisfying \( 2^{r} = 3 \).
\end{exercise}

% sol 1.2.2 -- TODO
\begin{solution}
    Assume there exists some rational $\frac{p}{q}$ such that $2^{\frac{p}{q}} = 3$ or equivalently, $2^p = 3^q$. We will show that no such rational exists by showing no value of $p$ can exist such that the above equation is satisfied.

    For the sake of the argument, let $p = 0$. Our equation now becomes $3^q = 1$ meaning $q = 0$ as well. Unfortunately, since $\frac{0}{0} \not \in \mathbb{Q}$, the pair $(0, 0)$ does not work and as such $p \not = 0$.

    Next, assume $p < 0$. Then, $2^p < 1$ which forces $3^q$ to be less than $1$ which means $q < 0$. However, notice that if there exists some pair $(p, q)$ such that $2^{\frac{p}{q}} = 3$ with both $p, q < 0$, then the pair $(-p, -q)$ must also be another solution to the equation as $2^{\frac{-p}{-q}} = 2^{\frac{p}{q}} = 3$. As such, if we prove that the equation $2^{\frac{p}{q}} = 3$ has no solutions for positive $p$, then that equation will also have no solutions for negative $p$.

    Finally then, we tackle the case when $p > 0$. $p > 0$ implies $p \in \mathbb{N}$ and as such it follows that $2^p$ is divisible by two and is even. Since the LHS of our equation ($2^p$) is divisible by 2, the RHS of our equation ($3^q$) must then also be divisible by 2 for the equality to hold. However, this is clearly impossible as an odd number times another odd number will be another odd number. As such, the RHS of our equation will never be even for positive integer values of $q$ and thus taken together with the above arguments, we conclude that no such $p$ can exist and as a result there exists no such rational $r$ such that $2^r = 3$.
\end{solution}




% ex 1.2.3
\begin{exercise}
    Decide which of the following represent true statements about the nature of sets. For any that are false, provide a specific example where the statement in question does not hold. \\

    (a) If \( A_{1} \supseteq A_{2} \supseteq A_3 \dots \) are all sets containing an infinite number of elements, then the intersection $\cap_{n = 1}^{\infty} A_n$ is infinite as well.

    (b) If \( A_{1} \supseteq A_{2} \supseteq A_3 \dots \) are all finite, nonempty sets of real numbers, then the intersection $\cap_{n = 1}^{\infty} A_n$ is finite and nonempty.

    (c) $A \cap (B \cup C) = (A \cap B) \cup C$. 

    (d) $A \cap (B \cap C) = (A \cap B) \cap C$.

    (e) $A \cap (B \cup C) = (A \cap B) \cup (A \cap C)$

\end{exercise}

% sol 1.2.3 -- TODO
\begin{solution}
    and their corresponding solutions.
\end{solution}


\end{document}
