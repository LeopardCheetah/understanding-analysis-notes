\documentclass[12pt]{article}

\usepackage{amsmath, amsthm, amssymb, amsfonts}
\usepackage{theoremref}
\usepackage{setspace}
\usepackage{geometry}

\usepackage{float}
\usepackage{hyperref}
\usepackage[utf8]{inputenc}
\usepackage[english]{babel}

\usepackage{titlesec}
\usepackage{thmtools}
\usepackage[dvipsnames]{xcolor}
\usepackage{tcolorbox}




%%%%%%%%%%%%%%%%%%%%%%%%%%
% horizontal bar
\newcommand{\hr}[1]{\rule{\linewidth}{#1}}
%%%%%%%%%%%%%%%%%%%%%%%%%%


%%%%%%%%%%%%%%%%%%%%%% used for changing size of \section and \subsection
\titleformat*{\section}{\huge\bfseries}
\titleformat*{\subsection}{\LARGE\bfseries}
% \titleformat*{\subsubsection}{\large\bfseries} - unnecessary
%%%%%%%%%%%%%%%%%%%%%%%%%%%%%%%%%%%%%%%%%%%


%%%%%%%%%%%%%%%%%%%%%%%%%%%%%%%%%%%%%%%%%%%%%%%%%
% thmtools - package for making \declare theorem
% https://ctan.math.illinois.edu/macros/latex/contrib/thmtools/doc/thmtools-manual.pdf
\declaretheoremstyle[]{box} % for space below see the 
% \tcolorboxenvironment{} line.


\declaretheorem[
    style = box,
    name = Exercise,
    numberwithin = subsection,
    thmbox = M
]{exercise}


% for solutions to the exercises
\declaretheorem[
    style = box,
    name = Solution,
    numbered = no
]{solution}



\tcolorboxenvironment{solution}{colframe=Green, colback=White, after skip=20pt}
\AtEndEnvironment{solution}{\qed}
%%%%%%%%%%%%%%%%%%%%%%%%%%%%%%%%%%%%%%%%%%%%%%%%%%%


\setstretch{1.1}
\geometry{
    top = 1in,
    margin = 1.12in, % wildstang plus one
    headheight = 12pt,
    headsep = 25pt,
    footskip = 30pt
}



% ------------------------------------------------------------------------------

\title{Understanding Analysis Exercises Solutions}
\author{LeopardCheetah}
\date{August 24, 2025}


\begin{document}

 
\maketitle
\newpage

% -----------------------------------------------------------------------------

% TODO -- edit this at a later date
% TODO -- get a vscode latex extension
\section*{Overview}

This document will be a compilation of all my (hopefully thorough) solutions to Stephen Abbott's \textit{Understanding Analysis}, a classic undergraduate textbook in real analysis. I hope that all solutions in this doc will leave the reader no doubt to the solution's correctness and will (hopefully) not use words such as ``trivial'', ``clearly'', or leave things up to the reader to figure out.

This document will \textbf{only} pertain to the solutions for Abbott's exercises; namely, this means that I assume the reader is concurrently reading Abbott's analysis textbook, and at some times I may cite theorems covered in the textbook that will not be reproduced here.\footnote{Maybe. Maybe I'll find it easier to put all theorems here so it's in a sense self contained but that's to be decided.}

Credits to Ulisse Mini and Jesse Li and for putting together the [hyperlink needed] solution manual that I based this off of; it's helped me a lot in my journey and at some points where I couldn't figure out solutions, I'll be citing their solutions. You can find their solution doc/repo [hyperlink needed] here.

Also credits to Ulisse for motivating me to make this document. I'm not sure it will be helpful to anyone other than myself, but it's good to have my solutions \LaTeX'd up and set in stone.

\newpage



\section{The Real Numbers}

% sec 1.1
\subsection{Discussion: The Irrationality of \(\sqrt{2}\) }
% some latex annoyance thing going on but wtv
% TODO -- troubleshoot so package hyperref doesn't get annoyed about this

% sec 1.2
\subsection{Some Preliminaries}

% ex 1.2.1 -- TODO: reformat so there is a bigger space between (a) and (b)
\begin{exercise}

    (a) Prove that \( \sqrt{3} \) is irrational. Does a similar argument work to show \( \sqrt{6} \) is irrational?

    (b) Where does the proof of Theorem 1.1.1\footnote{maybe add a hyper link to the thm 1.1.1 in the actual textbook?} break down if we try to use it to prove \( \sqrt{4} \) is irrational?
\end{exercise}
% sol 1.2.1 -- TODO
\begin{solution}
    solution to 1.2.1
\end{solution}


% ex 1.2.2
\begin{exercise}
    Show that there is no rational number \( r \) satisfying \( 2^{r} = 3 \).
\end{exercise}
% sol 1.2.2 -- TODO
\begin{solution}
    solution to 1.2.2
\end{solution}


% ex 1.2.3 -- TODO -- finish + fix
\begin{exercise}
    Decide which of the following represent true statements about the nature of sets. For any that are false, provide a specific example where the statement in question does not hold.

    (a) If \( A_{1} \subset A_{2} \) then \( x \).
\end{exercise}
% sol 1.2.3 -- TODO
\begin{solution}
    and their corresponding solutions.
\end{solution}





\end{document}











