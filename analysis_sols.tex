\documentclass[12pt]{article} 
\usepackage[utf8]{inputenc}
\usepackage[english]{babel}

\usepackage{amsmath, amsthm, amssymb, amsfonts}
\usepackage{theoremref}
\usepackage{setspace}
\usepackage{geometry}

\usepackage{float} 

\usepackage{titlesec}
\usepackage{thmtools}
\usepackage[dvipsnames]{xcolor}
\usepackage{tcolorbox}

\usepackage[unicode]{hyperref}
\hypersetup{
    colorlinks = true,
    filecolor = black,
    linkcolor = blue,
    linktoc = all,
    urlcolor = black,
    pdftitle = {my sols} % TO CHANGE LATER 
}


\usepackage{subfiles} % Best loaded last in the preamble



%%%%%%%%%%%%%%%%%%%%%%%%%%
% horizontal bar
\newcommand{\hr}[1]{\rule{\linewidth}{#1}}
%%%%%%%%%%%%%%%%%%%%%%%%%%


%%%%%%%%%%%%%%%%%%%%%% used for changing size of \section and \subsection
\titleformat*{\section}{\huge\bfseries}
\titleformat*{\subsection}{\LARGE\bfseries}
% \titleformat*{\subsubsection}{\large\bfseries} - unnecessary
%%%%%%%%%%%%%%%%%%%%%%%%%%%%%%%%%%%%%%%%%%%


%%%%%%%%%%%%%%%%%%%%%%%%%%%%%%%%%%%%%%%%%%%%%%%%%
% thmtools - package for making \declare theorem
% https://ctan.math.illinois.edu/macros/latex/contrib/thmtools/doc/thmtools-manual.pdf
\declaretheoremstyle[]{box} % for space below see the 
% \tcolorboxenvironment{} line.

\theoremstyle{definition}

% NOTE: for footnotes in here, use \footnotemark{} to initialize a footnote at the point and \footnotetext{text} to populate the footnote AFTER with text.
% unsure why this works better but it just does so
% also if you have multiple footnotes you'll have to manually sync each number up and use \footnotetext[number]{text} since that's just how it goes
\declaretheorem[
    style = box,
    name = Exercise,
    numberwithin = subsection,
    thmbox = {style=M,bodystyle=\normalfont}
]{exercise}


% for solutions to the exercises
% footnotes in this environment will appear DIRECTLY in the solution box, NOT in the main .tex file. 
% yeah, it's weird
\declaretheorem[
    style = box,
    name = Solution,
    numbered = no
]{solution}



\tcolorboxenvironment{solution}{colframe=Green, colback=White, after skip=40pt} % 40 pt = space after green box
\AtEndEnvironment{solution}{\qed}
%%%%%%%%%%%%%%%%%%%%%%%%%%%%%%%%%%%%%%%%%%%%%%%%%%%


\setstretch{1.1}
\geometry{
    top = 1in,
    margin = 0.75in, % wildstang plus one
    headheight = 12pt,
    headsep = 25pt,
    footskip = 30pt
}



% ------------------------------------------------------------------------------

\title{Understanding Analysis Exercises Solutions}
\author{Alex Z}
% started aug 24, 2025
\date{October 15, 2025} % change to most recent date


\begin{document}

 
\maketitle
\newpage

% -----------------------------------------------------------------------------

% TODO - Fix all the hyperlink needed sections
\section*{Overview}

This document will be a compilation of all my (hopefully thorough) solutions to Stephen Abbott's \textit{Understanding Analysis}, a classic undergraduate textbook in real analysis. I hope that all solutions in this doc will leave the reader no doubt to the solution's correctness and will (hopefully) not use words such as ``trivial'', ``clearly'', or leave things up to the reader to figure out.

This document will \textbf{only} pertain to the solutions for Abbott's exercises; namely, this means that I assume the reader is concurrently reading Abbott's analysis textbook, and at some times I may cite theorems covered in the textbook that will not be reproduced here.\footnote{Maybe. Maybe I'll find it easier to put all theorems here so it's in a sense self contained but that's to be decided.}

Credits to Ulisse Mini and Jesse Li and for putting together the [hyperlink needed] solution manual that I based this off of; it's helped me a lot in my journey and at some points where I couldn't figure out solutions, I'll be citing their solutions. You can find their solution doc/repo [hyperlink needed] here.

Also credits to Ulisse for motivating me to make this document. I'm not sure it will be helpful to anyone other than myself, but it's good to have my solutions \LaTeX'd up and set in stone.

\newpage

% -----------------------------------------------------------------------------

\tableofcontents
\newpage 

% -----------------------------------------------------------------------------

% actual content is below.


\section{The Real Numbers}

% sec 1.1
% somehow this fixes everything
% gets rid of warnings and everything
% why? who knows
\subsection{Discussion: The Irrationality of \texorpdfstring{\( \sqrt{2} \)}{Root 2}} 



%%%%%%%%%%%%%%%%%%%%%%%%


% sec 1.2
\subsection{Some Preliminaries}
\subfile{ch1/1-2}



  


\end{document}











