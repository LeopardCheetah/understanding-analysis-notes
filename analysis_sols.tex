\documentclass[12pt]{article}

\usepackage{amsmath, amsthm, amssymb, amsfonts}
\usepackage{theoremref}
\usepackage{setspace}
\usepackage{geometry}

\usepackage{float}
\usepackage{hyperref}
\usepackage[utf8]{inputenc}
\usepackage[english]{babel}

\usepackage{titlesec}
\usepackage{thmtools}
\usepackage[dvipsnames]{xcolor}
\usepackage{tcolorbox}




%%%%%%%%%%%%%%%%%%%%%%%%%%
% horizontal bar
\newcommand{\hr}[1]{\rule{\linewidth}{#1}}
%%%%%%%%%%%%%%%%%%%%%%%%%%


%%%%%%%%%%%%%%%%%%%%%% used for changing size of \section and \subsection
\titleformat*{\section}{\huge\bfseries}
\titleformat*{\subsection}{\LARGE\bfseries}
% \titleformat*{\subsubsection}{\large\bfseries} - unnecessary
%%%%%%%%%%%%%%%%%%%%%%%%%%%%%%%%%%%%%%%%%%%


%%%%%%%%%%%%%%%%%%%%%%%%%%%%%%%%%%%%%%%%%%%%%%%%%
% thmtools - package for making \declare theorem
% https://ctan.math.illinois.edu/macros/latex/contrib/thmtools/doc/thmtools-manual.pdf
\declaretheoremstyle[]{box} % for space below see the 
% \tcolorboxenvironment{} line.


\declaretheorem[
    style = box,
    name = Exercise,
    numberwithin = subsection,
    thmbox = M
]{exercise}


% for solutions to the exercises
\declaretheorem[
    style = box,
    name = Solution,
    numbered = no
]{solution}



\tcolorboxenvironment{solution}{colframe=Green, colback=White, after skip=20pt}
\AtEndEnvironment{solution}{\qed}
%%%%%%%%%%%%%%%%%%%%%%%%%%%%%%%%%%%%%%%%%%%%%%%%%%%


\setstretch{1.1}
\geometry{
    top = 1in,
    margin = 1.12in, % wildstang plus one
    headheight = 12pt,
    headsep = 25pt,
    footskip = 30pt
}



% ------------------------------------------------------------------------------

\title{Understanding Analysis (Abbott) Solutions}
\author{LeopardCheetah}
\date{August 24, 2025}


\begin{document}

 
\maketitle
\newpage

% ------------------------------------------------------------------------------

\section*{Overview}

% things to talk about:
% - how document is structured
% - assumption that all understanding analysis book has been read -- so i can cite things (thm 2.3.4) directly if necessary
% - im gonna try not to use the words "trivial" or "clearly" in here and try to explain everything in depth
% - credits to those other guys for first having me do this but also for their sols which definitely helped me along the way -- i got stuck on some of them so i'll refer back to them for their solution.

some overview material will be provided here later.

\newpage



\section{Logarithms}
Recall that a logarithm is a function \( g(x) \) (of base \( e \)) such that \( g(e^x) = x \) and \( e^{g(x)} = x \).


\subsection{Properties}
These are 5 main properties of logarithms:

\begin{exercise}
    this is some exercise.
\end{exercise}

\begin{solution}
    and this is the solution.
\end{solution}

\begin{exercise}
    here are some more exercises
\end{exercise}
\begin{exercise}
    and even more exercises.
\end{exercise}
\begin{solution}
    and their corresponding solutions.
\end{solution}





\end{document}











